\documentclass[10pt,twocolumn]{article} 

\usepackage{oxycomps} % use the main oxycomps style file

\bibliography{references}

\pdfinfo{
    /Title (Why adoption of Edtech in Schools is Unethical)
    /Author (Kathy Liu)
}

\title{Why adoption of Edtech in Schools is Unethical}

\author{Kathy Liu}
\affiliation{Occidental College}
\email{kliu4@oxy.edu}

\begin{document}

\maketitle

\section{Introduction}
Educational technology, or Edtech, is now highly integrated within schools at all levels. General sentiments toward edtech are positive, with many feeling it will be the “solution” to “fix” education. However, this paper will explore several reasons why the adoption of Edtech in schools is unethical, such as causing asymmetrical shifts in power, negative economic effects, and concerns regarding data privacy.

\section{Background}
Since the rise of the internet and PCs, the traditional classroom model of education has been becoming increasingly virtual. In the more recent years, emerging technologies such as gamification, augmented reality, new educational applications, and the Internet of Things will continue the upheaval of traditional education \cite{zain_2021}. Such applications include management systems such as Canvas, ClassDojo, and Google Classroom, assessment systems such as Pearson’s MathXL, and online alternatives for physical textbooks. Thus, educational institutions have adopted technologies for several of its major functions. Starting in middle school, technology becomes relevant for students to participate and learn, and by high school and college, it is often the principal means. 

This trend toward E-learning, or using computers and the Internet as a core component of learning, has been hugely exacerbated by the forced shift to remote learning during the height of the COVID-19 pandemic. It is also important to note that E-learning may or may not be based in a physical classroom, whereas distance learning adopted during the pandemic is remote and utilizes E-learning. Results from the EU Public consultation from late 2020 showed 60\% of respondents had not used distance and online learning before the crisis, and 95\% consider the crisis marks a point of no return for how technology is used in education and training \cite{2020_EU_public_consultation}. Coming out of the pandemic, E-learning is more widely accepted than ever. 
 
On the surface, it may appear that educational technology, or Edtech, is the key to bringing forth a utopia of equitable, quality education. Those who favor E-learning laud the fairness it gives students; unlike a human teacher who may have bias toward different kinds of students, a preprogrammed robot would teach and grade the same toward every student. Additionally, AI algorithms incorporated into many Edtech products are capable of offering unlimited attention to each student. This scalability is also what makes E-learning very appealing to education institutions, as it can be much more cost effective than traditional means. E-learning may also offer education on-demand, so that students may build self-motivation to complete schoolwork outside of the classroom, at their own pace and convenience \cite{hamid_2001}. However, as with all trends, we must be prudent to look beneath the surface and critically examine the true effects of automating education.
    
\section{How E-learning Shifts Power}
\subsection{Enforcing fixed pedagogy}
If Edtech becomes popularly adopted in schools, the tech companies behind the products will undoubtedly come out as one of the winners. The current Edtech industry is already huge; it global market is valued at hundreds of billions of dollars, and includes education-specific companies, such as Chegg Inc., as well as most of the big-name multinational tech companies such Microsoft and Google. In addition to gaining revenue, these companies will gain increasing influence on the construction of classrooms. The companies inevitably code their assumptions or ideologies of how education could or should operate into their digital products, that schools then become dependent on \cite{williamson_2017}. By using these products, educational institutions and individual teachers are giving up some freedom in organizing their classes. For instance, teachers may enjoy the convenience of using a platform such as Moodle to consolidate the transactions like turning in assignments, but in some way, they must also adapt their transactions to fit Moodle’s encoded paradigms of the what an assignment is, what form they come in, how students turn them in, students’ experience of turning assignments in, and much more. 

Currently, some educational institutions face pressure to “modernize” by offering more online courses and components, but once they start adopting these technologies, they “may then be reshaping how schools and universities function, ultimately configuring new kinds of cloud classrooms, could campuses, and cloud educators that are interdependent with interoperable infrastructures and marketplaces of platform integrations” \cite{williamson_2017}. Without proper precaution, E-learning may cause convergence of education into hegemonic or otherwise fixed conceptions of the classroom, dependent on tech companies’ online products. 
 

\subsection{Accessibility}
Edtech will also affect groups of teachers and students differently. As much as Edtech has the potential to create equity in education, it could also further deepen inequity. For example, to successfully use Edtech products, teachers and students will need internet access, a compatible device, and possess digital literacy at a minimum. Access to all three of these requirements are also tied to one’s socioeconomic status. Students of low socioeconomic status are already widely regarded as educationally disadvantaged \cite{walpole_2003}. Additionally, the successes and attention of Edtech are largely contained within developed nations \cite{bulathwela_2021}. Thus in application, the benefits of Edtech that make it seem appealing would likely be most accessible to those that are already educationally privileged.  

\section{Negative Economic Effects}
\subsection{Displacement of Workers}
In many cases, Edtech is replacing the work of actual humans, rather than merely assisting or augmenting their jobs. A straightforward example of this would be student graders, whose roles have become a standard part of many undergraduate-level classrooms. There are countless Edtech products on the market with automatic grading capabilities, such as online homework and test programs that can produce instantaneous feedback on the correctness of an answer, if given the answer key. Some academic subjects may not inherently have discrete correct answers, such as most questions that require written explanation, and literature in general. However, grading even these types of questions would be made much less time-consuming with automatic plagiarism detectors built into online submission systems. Altogether, the amount of work by a human required to grade, and thus, the number of jobs available, decreases significantly with the introduction of automated grading in Edtech products. 

Some may argue that engineers are needed to create the products, so the total number of jobs may not actually be different than before. But crucially, these engineering jobs are not a relevant substitute to those actually being displaced–undergraduate students. Recent investigation into the undergraduate population within the US revealed that the majority of students enrolling in postsecondary education are low-income \cite{fountain_2019}. Therefore, the dissolution of jobs immediately available to undergraduate students has a profound impact on the livelihood of the displaced workers. 


\subsection{Commodification of Higher Education}
As with any company, the firms of the Edtech industry are profit-maximizing institutions woven into a greater market economy, and so one must ask if they really value the public good over their bottom line. As discussed previously, Edtech has the power to define and perpetuate fixed constructions of education, with these fixed constructions perhaps being more self-serving than most realize. It is against the Edtech industry’s capitalist interests to make education truly democratized and financially accessible. Instead, these firms rely on converting the public good–the human right–of education into a private one: “Public knowledge is fast becoming the site of private accumulation for the Edtech industry. Knowledge is reduced in ‘content,’ for Edtech to copyright and sell” \cite{mirrlees_2019}. While Edtech is praised for its cost effectiveness in delivering course content to more students, the price of higher education and student debt continue to climb each year\cite{brundick_2019} \cite{brown_2015}. In a higher education landscape that is already highly commodified, the Edtech industry feeds into the gatekeeping of knowledge from those who cannot pay. This commodification and commercialization of instruction is the very remedy that is so trumpeted as the “savior” of education \cite{noble_1998}. Meanwhile, higher education institutions will adopt Edtech to increase profit margins, but without lowering tuition. The result is a covert reduction of public investment in education, with some actors deepening their pockets, and the students bearing the consequences.

\section{Data Privacy}
Coveted features in Edtech, such as learning analytics, personalized learning, and even emotion sensing all utilize some form of artificial intelligence. Creating input for these AI algorithms requires what some call the datafication of education \cite{williamson_2017}. Each score, click, keystroke, and tab switch, from students are tracked and collected to create data profiles of each student. These profiles are then used as insight into a student’s thought process, behavioral patterns, and learning style. On the surface, these abilities sound like those that any good teacher should have, but in the process, students are being reduced to data points in order to capture and manipulate their psychological state. In an critical examination of ClassDojo, a popular Edtech platform that tracks student behavior, by \textcite{manolev_2019}, the “reduction of students to data points based on the performance of behavior facilitates data-driven techniques of governance that function through the classification, ranking and comparison of students.” In this way, datafication is a central technique in creating technologies of control and surveillance. Furthermore, Edtech companies are able to collect this data from students to develop, and thus further profit from, their AI algorithms. According to \textcite{williamson_2017}, “Few students realize that their computer-based courses are often thinly-veiled field trials for product and market development, that while they are studying their courses, their courses are studying them.” The automation of education entails datafication of students, which causes each action of students to be much closer tracked and recorded than any form of traditional education. With this information, Edtech products are then able to nudge students into compliance with “ideal” behavior, while expanding their influence by using students as a mine for learning analytics.

\section{Conclusion}
Despite all the hype surrounding E-learning, it may come with greater ethical concerns than promised. Instead of Edtech being a tool to increase accessibility to education, it has just as much potential to reinforce existing inequities. Companies producing Edtech products also place themselves in positions of expanding power at the cost of teachers and students; by using Edtech, teachers give up decisive power over their pedagogy, and students give up their data and autonomy to Edtech. There are also concerns for the economic impact of the displacement of some workers in education, and blindly following trends such as greater neoliberal efficiency gains. All of these are ethical issues that must be appropriately addressed by the Edtech industry, the educational institutions that purchase them, and the teachers and students that use them. 

\printbibliography 

\end{document}
